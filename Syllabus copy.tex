% Don't touch this %%%%%%%%%%%%%%%%%%%%%%%%%%%%%%%%%%%%%%%%%%%
\documentclass[11pt]{article}
\usepackage{fullpage}
\usepackage[left=1in,top=1in,right=1in,bottom=1in,headheight=3ex,headsep=3ex]{geometry}
\usepackage{graphicx}
\usepackage{float}

\newcommand{\blankline}{\quad\pagebreak[2]}
%%%%%%%%%%%%%%%%%%%%%%%%%%%%%%%%%%%%%%%%%%%%%%%%%%%%%%%%%%%%%%

% Modify Course title, instructor name, semester here %%%%%%%%

\title{Ecology 98: Study in Bears}
\author{Yifan Ding, Chan Gao, Jimmy Nguyen}
\date{Fall, 2019}

%%%%%%%%%%%%%%%%%%%%%%%%%%%%%%%%%%%%%%%%%%%%%%%%%%%%%%%%%%%%%%

% Don't touch this %%%%%%%%%%%%%%%%%%%%%%%%%%%%%%%%%%%%%%%%%%%
\usepackage[sc]{mathpazo}
\linespread{1.05} % Palatino needs more leading (space between lines)
\usepackage[T1]{fontenc}
\usepackage[mmddyyyy]{datetime}% http://ctan.org/pkg/datetime
\usepackage{advdate}% http://ctan.org/pkg/advdate
\newdateformat{syldate}{\twodigit{\THEMONTH}/\twodigit{\THEDAY}}
\newsavebox{\MONDAY}\savebox{\MONDAY}{Mon}% Mon
\newcommand{\week}[1]{%
%  \cleardate{mydate}% Clear date
% \newdate{mydate}{\the\day}{\the\month}{\the\year}% Store date
  \paragraph*{\kern-2ex\quad #1, \syldate{\today} - \AdvanceDate[4]\syldate{\today}:}% Set heading  \quad #1
%  \setbox1=\hbox{\shortdayofweekname{\getdateday{mydate}}{\getdatemonth{mydate}}{\getdateyear{mydate}}}%
  \ifdim\wd1=\wd\MONDAY
    \AdvanceDate[7]
  \else
    \AdvanceDate[7]
  \fi%
}
\usepackage{setspace}
\usepackage{multicol}
%\usepackage{indentfirst}
\usepackage{fancyhdr,lastpage}
\usepackage{url}
\pagestyle{fancy}
\usepackage{hyperref}
\usepackage{lastpage}
\usepackage{amsmath}
\usepackage{layout}

\lhead{}
\chead{}
%%%%%%%%%%%%%%%%%%%%%%%%%%%%%%%%%%%%%%%%%%%%%%%%%%%%%%%%%%%%%%

% Modify header here %%%%%%%%%%%%%%%%%%%%%%%%%%%%%%%%%%%%%%%%%
\rhead{\footnotesize Text in header}

%%%%%%%%%%%%%%%%%%%%%%%%%%%%%%%%%%%%%%%%%%%%%%%%%%%%%%%%%%%%%%
% Don't touch this %%%%%%%%%%%%%%%%%%%%%%%%%%%%%%%%%%%%%%%%%%%
\lfoot{}
\cfoot{\small \thepage/\pageref*{LastPage}}
\rfoot{}

\usepackage{array, xcolor}
\usepackage{color,hyperref}
\definecolor{clemsonorange}{HTML}{EA6A20}
\hypersetup{colorlinks,breaklinks,linkcolor=clemsonorange,urlcolor=clemsonorange,anchorcolor=clemsonorange,citecolor=black}

\begin{document}

\maketitle

\blankline

\begin{tabular*}{.93\textwidth}{@{\extracolsep{\fill}}lr}

%%%%%%%%%%%%%%%%%%%%%%%%%%%%%%%%%%%%%%%%%%%%%%%%%%%%%%%%%%%%%%

% Modify information %%%%%%%%%%%%%%%%%%%%%%%%%%%%%%%%%%%%%%%%%
E-mail: \texttt{yifan.ding@berkeley.edu} & Web: \href{www.ucberkeley.edu/~yifan.ding}{\tt\bf www.ucberkeley.edu/~yifan.ding}  \\

 Office Hours: M 10-11:45am  &  Class Hours: T 3-4pm \\

 Office: 3056 VLSB & Class Room: 2050 VLSB \\
 & \\
Lab Room: 203 LeConte & Lab Hours: W 3-5pm \\
&  \\
\hline
\end{tabular*}

\vspace{5 mm}

% First Section %%%%%%%%%%%%%%%%%%%%%%%%%%%%%%%%%%%%%%%%%%%%

\section*{Course Description}

Welcome to the Bears decal. Over the course of the semester, you will learn about the different species of bears ranging from brown bears, polar bears, giant pandas, black bears, wolverine, cave bear, ringed seal, etc. Surprised by some of the bears mentioned above? We are too. The goal for this course is for you to gain fun knowledge on bears: their living habits, diets, the geographical areas that they span, and to bring awareness to some species that are at risk of being endangered. We hope that you'll enjoy learning these bear facts! 

% Second Section %%%%%%%%%%%%%%%%%%%%%%%%%%%%%%%%%%%%%%%%%%%

\section*{Required Materials}

\begin{itemize}
\item Course notes available on bCourses. Course will require sufficient upper body strength to bench press 250lbs, squat 360lbs, and deadlift 400lbs. 
\end{itemize}


% Fourth Section %%%%%%%%%%%%%%%%%%%%%%%%%%%%%%%%%%%%%%%%%%%

\section*{Course Objectives}
Successful students will:
\begin{enumerate}
\item gain an understanding about the nature of bears 
\item strengthen the bond between themselves and their mascot
\item establish evidence that Golden Bears are the best bears
\item develop their ability to engage in hand to hand combat with bears
\item understand the place of bears in the ecosystem, and the vital role they play in bringing down trees 
\end{enumerate}

% Fifth Section %%%%%%%%%%%%%%%%%%%%%%%%%%%%%%%%%%%%%%%%%%%

\section*{Course Structure}

\subsection*{Lecture}

Class will begin with presentations about bears, after which homework will be given out. After which, the lecturer will cover the week's content for the remaining duration of time. 

\subsubsection*{Labs}

Labs will begin with 30 minutes of warm up, while your instructor of choice trains you in the art of wrestling bears at their weak spots. Remaining lab time will be spent wrestling with said bears to gauge their strengths. Each week, a different species of bear will be presented to the students. Careful of the Grizzlys! 

\subsection*{Assessments}

Assessments will be made at the end of each lab on the correct techniques used in surviving a wrestling match with a bear. Go Bears!

\subsubsection*{Final Exam and Class Project}

Final examination will be in 150 Wheeler on December 18th (so don't go anywhere on winter break). Final lab will include the development of a personalized technique to use one's inherent strengths to survive longer in a bear fighting cage than previously. 

\subsection*{Grading Policy}
The typical UC Berkeley grading scale will be used. I reserve the right to curve the scale dependent on overall class scores at the end of the semester. Any curve will only ever make it easier to obtain a certain letter grade. The grade will count the assessments using the following proportions:
\begin{itemize}
	\item \underline{\textbf{10\%}} of your grade will be determined by the class final (15\% each).
	\item \underline{\textbf{5\%}} of your grade will be determined by how well you can wrestle and survive in the fighting pit
	\item \underline{\textbf{85\%}} Participation in lecture, and the professor's ability to recognize your face in a crowd of C- students. 
\end{itemize}

\section*{Course Policies}

\subsection*{During Class}
\footnotesize{I understand that the electronic recording of notes will be important for class and so computers will be allowed in class. Please refrain from using computers for anything but activities related to the class. Phones are prohibited as they are rarely useful for anything in the course. Eating and drinking are allowed in class but please refrain from it affecting the course. Try not to eat your lunch in class as the classes are typically active.}

\subsection*{Attendance Policy}
\footnotesize{For complete attendance and excused absence policies, please see http://policies.ucberkeley.edu/regulation/reg-02-20-03. Attendance is expected in all lecture and lab sections. Valid excuses for absence will be accepted before class. In extenuating circumstances, valid excuses with proof will be accepted after class. For every class missed the participation grade will be dropped 1 point.}

\subsection*{Policies on Incomplete Grades and Late Assignments}
\footnotesize{If an extended deadline is not authorized by the instructor or department, an unfinished incomplete grade will automatically change to an F after either (a) the end of the next regular semester in which the student is enrolled (not including summer sessions), or (b) the end of 12 months if the student is not enrolled, whichever is shorter. Incompletes that change to F will count as an attempted course on transcripts. The burden of fulfilling an incomplete grade is the responsibility of the student. The university policy on incomplete grades is located at http://policies.ucberkeley.edu/regulation/reg-02-50-3.}

\footnotesize{Late assignments will be accepted for no penalty if a valid excuse is communicated to the instructor before the deadline. After the deadline, assignments will be accepted for a 50\% deduction to the score up to 2 days after the deadline. After this any assignments handed in will be given 0.}

\subsection*{Academic Integrity and Honesty}
\footnotesize{Students are required to comply with the university policy on academic integrity found in the Code of Student Conduct found at http://policies.ucberkeley.edu/policy/pol-11-35-01. Don't cheat. Don't be that guy. Yes, you. You know exactly what I'm talking about. See http://policies.ucberkeley.edu/policy/pol-11-35-01 for a detailed explanation of academic honesty.}

\subsection*{Accommodations for Disabilities}
\footnotesize{Reasonable accommodations will be made for students with verifiable disabilities. In order to take advantage of available accommodations, students must register with the Disability Services Office at Suite 2221, Student Health Center, Campus Box 7509, 919-515-7653. For more information on the University of California's policy on working with students with disabilities, please see the Academic Accommodations for Students with Disabilities Regulation (REG02.20.01) (https://policies.ucberkeley.edu/regulation/reg-02-20-01/).
Non-Discrimination Policy University of California provides equality of opportunity in education and employment for all students and employees. Accordingly, University of California affirms its commitment to maintain a work environment for all employees and an academic environment for all students that is free from all forms of discrimination.}

\footnotesize{Discrimination based on race, color, religion, creed, sex, national origin, age, disability, veteran status, or sexual orientation is a violation of state and federal law and/or University of California policy and will not be tolerated. Harassment of any person (either in the form of quid pro quo or creation of a hostile environment) based on race, color, religion, creed, sex, national origin, age, disability, veteran status, or sexual orientation also is a violation of state and federal law and/or University of California policy and will not be tolerated. Retaliation against any person who complains about discrimination is also prohibited. University of California's policies and regulations covering discrimination, harassment, and retaliation may be accessed at \href{http://policies.ucberkeley.edu/policy/pol-04-25-05} or  \href{http://www.ucberkeley.edu/equal_op/}. Any person who feels that he or she has been the subject of prohibited discrimination, harassment, or retaliation should contact the Office for Equal Opportunity (OEO) at 919-515-3148.}

% Course Schedule %%%%%%%%%%%%%%%%%%%%%%%%%%%%%%%%%%%%%%%%%%%

\newpage
\section*{Schedule and weekly learning goals}

The schedule is tentative and subject to change. The learning goals below should be viewed as the key concepts you should grasp after each week, and also as a study guide before each exam, and at the end of the semester. Each exam will test on the material that was taught up until 1 week prior to the exam (i.e. vorticity will not be tested until exam 2). The applications in the second half of the semester tend to build on the concepts in the first half of the semester though, so it is still important to at least review those concepts throughout the semester.

% Set first date of the semester (for some reason this is a week before what comes up, but that's easy to get around)
\SetDate[01/01/2018]
\week{Week 01} Origins of Bears
\begin{itemize}
\item Introduction: Your Instructor 
\item Lecture: History of Bears
\item Reading Assignment: How to defend yourself from Grizzly Bears 101 - James Arthur 
\end{itemize}

\week{Week 02} Survival and Environment of Bears
\begin{itemize}
\item Group Discussion Activity: What would you do in a situation where you saw a bear in the woods ?
\item Lecture: Habitats of Different Bears
\item First Writing Assignment: 3-Page Essay on who would win in a fight ? Pandas vs Polar Bears
\end{itemize}

\week{Week 03} What's inside a Bear?
\begin{itemize}
\item Lecture: Bears' Anatomy
\item Activity: In Class Handout - Labeling parts of Bears
\item Post-Quiz: Anatomy and History of Bears
\end{itemize}

\week{Week 04} Guest Speaker
\begin{itemize}
\item Introduce Group Project: A Final report and presentation on selected type of bears
\item Lecture: Visitor/Guest Speaker from San Francisco Zoo on confronting a Bear
\item Homework: Proposal for research report on type of bears
\end{itemize}

\week{Week 05} HOLIDAY: NO LECTURE
\begin{itemize}
\item Homework: Work on group research report
\end{itemize}

\week{Week 06} Research Report Due
\begin{itemize}
\item Students' Final Presentation of Research Report
\end{itemize}


\week{Week 7} Final Exam: 2 Hrs
\begin{itemize}
\item Final consists of the following topics:
 Type of Bears, History of Bears, Anatomy of Bears, Habitat of Bears, and Self-Defense Against Bears
 \item GO BEARS !!!!!
\end{itemize}

\end{document}


